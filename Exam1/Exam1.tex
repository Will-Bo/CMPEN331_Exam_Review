\documentclass{article}

\usepackage{amsmath, amsfonts}
\usepackage{geometry}

\title{CMPEN 331 Exam 1 Review}
\author{Will Bochnowicz}

\usepackage{imakeidx}
%\makeindex
\makeindex[columns=2, title=Index, intoc]

\begin{document}

\maketitle

\tableofcontents

\section{Memory}

\subsection{Memory Layout}

Memory is where a computer stores information. To be accessible, each memory location must have an address\index{memory address}. The memory space is $2^k$ locations wide, which is called the address space. For example, a computer with 32 address bits can have at most $2^{32} \text{bits} = 4G$ of memory. Our computers are \underline{byte addressable}\index{memory address!byte-addressable}, where each byte location (1, 2, 3, \ldots) refers to a group of 8 bits in memory that are treated as a single unit. Because our computers have 32 bit words, word locations are at 0, 4, 8, \ldots. We like to align words to make them easier to access -- each word starts at a byte that is a multiple of the number of bytes in a word.

MIPS is Big-Endian\index{endian-ness}: This means that words are arranged such that lower addresses are the more significant bytes of a word. For example, 0xDEADBEEF, in a big-endian system, would be stored as DE AD BE EF in a single word, while a Little-Endian system could store this word as EF BE AD DE.

There are two memory operations -- read and write. 

A program's memory is laid out as a stack -- lower addresses (around 0) are reserved, then there are \texttt{Text}, \texttt{Static Data}, \texttt{Dynamic Data} (Heap), and \texttt{Stack} areas in memory. The Stack is located at the highest memory address (\texttt{0xffffffff}) and grows down, while the heap is located lower than the stack and grows up. These two regions can collide, causing memory errors, is either is too large. 

\section{MIPS Instructions}
%% What is is, description of how it works. The meaning of each of the fields (Maybe more specialized in each subsection)

Assembly is the language of the computer. These are basic operations that are easily converted (by the \underline{assembler}\index{assembler}) into \underline{machine code}\index{machine code}: the 1s and 0s that computers actually use to compute numbers. It is one of a number of \underline{Instruction Set Architectures}\index{ISA}, formal definitions of what a processor can and can't do, which have slightly varying \underline{Instruction Sets} per processor model. For example, ARMv7-m is an \underline{Instruction Set Architecture}, upon which the Cortex-M processors have conforming \underline{Instruction Sets} despite having slightly different implementations. 

MIPS is a family of Instruction Set Architectures. They are RISC\index{RISC} based, which means that they are relatively restricted in size and can only do operations on the contents of registers. The contents of memory must be moved into a register before it can be acted upon (load/store architecture\index{RISC!load/store architecture}). There are a number of ways that MIPS instructions\index{Instructions}, which are actual commands to the processor, are represented in binary. For the sake of this course, we only consider 32-bit\index{Word Size} MIPS architectures, which means that instructions are only 32 bits long. 

Sidenote: Other instruction set paradigms, such as Complex Instruction Set Computers (CISC) exist. CISC can have multi-word instructions and allow operands directly from memory. 

A register\index{register} is a small storage medium that is located directly in the CPU. It is the fastest form of memory, but has extremely limited size due to its location within the CPU. MIPS has 32 general purpose registers, as well as some others (LO, HI, PC, floating point registers) that serve specific purposes\index{register!MIPS register count}. 

Some MIPS instructions are so-called ``pseudoinstructions''\index{pseudoinstruction}, as they are not actual instructions in the MIPS ISA. Rather, they are shorthand for ease of programming and are expanded into actual instructions by the assembler. For example, the \texttt{blt \$s1, \$s2, Label} instruction actually expands to two instructions: \texttt{slt \$at, \$s1, \$s2} and \texttt{bne \$at, \$zero, Label}, where \texttt{\$at} is a register that is reserved for the assembler to do operations of this sort (side-note: \texttt{\$at} is one of the 32 general-purpose registers, despite being reserved by the assembler). 


\subsection{R-Type}\label{RType}\index{Instructions!R-Type}
R-Type instructions represent instructions that act on three registers. These registers are labeled as the Source, Target, and Destination

Examples of R-Type instructions: \texttt{add}, \texttt{sub}, \texttt{mul}
%% Diagram of representation, number of bits

\begin{tabular}{|c|c|c|c|c|c|}
    \hline
    op & rs & rt & rd & shamt & func \\
    \hline
    6 & 5 & 5 & 5 & 5 & 6 \\ 
    \hline
\end{tabular}

op: ``Operation'' -- The basic operation of the instruction, also called ``opcode''. 6 bits (just happened to be leftover from dividing 32 bits between 4 5-bit fields and 2 other fields).

rs: ``Resister Source'' -- the first operand of the instruction. 5 bits, as there are 32 general registers. 

rt: ``Register Target'' -- the second operand of the instruction. This can occasionally serve as the destination in other types of operations. 5 bits, as there are 32 general registers. 

rd: ``Register Destination'' -- the third operand of the instruction. This is where the result of the operation is stored. 5 bits, as there are 32 general registers. 

shamt: ``Shift Amount'' -- The amount that the target register is shifted, used for sll and srl operations (where the source register is unused). This field is zero for most arithmetic operations. 5 bits, in order to access all 32 bits in a word.

func: ``Function Code'' -- Specifies the variant of the Operation that is used. For most R-Type, this contains the entire operation (``op'' is zero for add, sub, mult, div, etc). 6 bits.

\subsubsection{Problems}\label{RTypeProblems}

\begin{equation}\label{RType1}
\begin{aligned}
    Q: & \text{Write the binary representation of the \texttt{add} command.}\\
    & \text{Note that \texttt{add}'s \texttt{op} and \texttt{func} fields are 000000 and 100000, respectively.}\\
    & \text{Also assume that for } a = b + c \text{, a is in register 1, b is in register 2, and c is in register 3.}
\end{aligned}
\end{equation}

\subsection{I-Type}\label{IType}\index{Instructions!I-Type}

I-Type instructions are instructions that work on immediates\index{immediate, data-type}, which are constants in memory. These instructions have extra space to store the binary representation of a number directly in their instructions, rather than in a source, target, or destination register. 

Instructions of I-Type instructions are all Load and Store instructions (lw/sw: \texttt{op base rt offset}), \texttt{addi} and \texttt{subi}, and Branch instructions (beq/bne: \texttt{op rs rt offset}, where offset is where you jump relative to PC\index{Addressing Modes!PC-Relative Addressing})

\begin{tabular}{| c | c | c | c |}
    \hline
        op & rs & rt & immediate \\
        \hline
        6 & 5 & 5 & 16 \\
    \hline
\end{tabular}

op: ``Operation'' -- The basic operation of the instruction, also called ``opcode''. 6 bits.

rs: ``Resister Source'' -- The first operand of the instruction. 5 bits, as there are 32 general registers. 

rt: ``Register Target'' -- The second operand of the instruction. This can occasionally serve as the destination in other types of operations. 5 bits, as there are 32 general registers. 

immediate: ``Immediate'' -- The constant immediate value that is being used in the operation. 16 bits can store up to the value 65,536.

\subsubsection{Problems}\label{ITypeProblems}

\begin{equation}\label{IType1}
\begin{aligned}
    Q: & \text{Write the MIPS assembly instruction for the following I-Type binary instruction:}\\
    & 00010101000010011010101010101010\\
    & \text{You can look up a table of MIPS commands in binary and the addresses of registers.}
\end{aligned}
\end{equation}


\subsection{J-Type}\label{JType}\index{Instructions!J-Type}

J-Type instructions are used exclusively for jump instructions. They feature the most space of any instruction for immediates, which are used to represent the address that is being jumped to. While only 26 bits, these bits can be expanded to the full 32 bit word size by concatenating the highest 4 bits of PC to the target address (somewhat limiting the range of this jump), and concatenating 00 to the end of the instruction, as instructions are always aligned on 32-bit words. 

\begin{tabular}{| c | c |}
    \hline
        op & target address \\
        \hline
        6 & 26 \\
    \hline
\end{tabular}

op: ``Operation'' -- The basic operation of the instruction, also called ``opcode''. 6 bits.

target address: ``Target Address'' -- An immediate that represents the instruction number that is to be jumped to. 

\subsubsection{Problems}\label{JTypeProblems}



\section{Addressing Modes}\label{Addressing}\index{Addressing Modes}

There are multiple ways that memory is addressed for various operations in MIPS assembly. 


\subsection{Register Addressing}\label{Register Addressing}\index{Addressing Modes!Register Addressing}

Register Addressing is a form of \underline{direct addressing} -- the value inside a register is used as a memory address. The \texttt{jr} and \texttt{jalr} use this form of addressing. 

Algebraic operations, such as \texttt{add}, \texttt{sub}, etc, all use this method of addressing. 


\subsection{Immediate Addressing}\label{Immediate Addressing}\index{Addressing Modes!Immediate Addressing}

An immediate value that is inserted directly into the instruction is used as a memory address. Limited to 16 bits, as that is the size of the Immediate field in I-Type instructions. The \texttt{addi} and related commands use immediate addressing, as the value of the immediate is baked into the command itself. 


\subsection{Base Addressing}\label{Base Addressing}\index{Addressing Modes!Base Addressing}

Base addressing is where you reference a location in memory based on an offset from a memory location that is stored in a register. Examples of this include the \texttt{lw} command. Once again, due to being an I-Type instruction, the offset values are limited to 16 bits. 


\subsection{PC-Relative Addressing}\label{PC-Relative Addressing}\index{Addressing Modes!PC-Relative Addressing}

PC-Relative Addressing is used in the \texttt{beq} and \texttt{bne} instructions. In this mode, the memory location of data or an instruction is specified using an offset from the current PC value. Once again, this type of addressing is used for I-Type instructions where the offset is stored as an immediate constant. 

One advantage of this method is that it works regardless of where the code is located in memory. While some other addressing modes can produce code that point directly to a specific point in memory, that form of addressing depends on those instructions being located in that specific memory location. PC-Relative Addressing creates machine code that acts independently of what specific location in memory the instructions are stored -- the offsets are always the same between instructions because word size is known (can be implemented at compile time), but exact locations in memory are only known at link time.


\subsection{Jump Addressing / Pseudo-direct Addressing}\label{Jump Addressing}\index{Addressing Modes!Jump Addressing}

Using J-Type instructions, this type of addressing is only used for \texttt{j} and \texttt{jal} commands. Expansion to a full address is done by concatenating the highest 4 bits of PC to the 26 bits of the target address (somewhat limiting the range of this jump), then concatenating 00 to the end of the instruction, as instructions are always aligned on 32-bit words. (Copied from J-Type explanation above).

The limitations of this addressing type come from using the upper 4 bits of PC -- this limits the jump to the current 256MB of code, which is only $\frac{1}{16}$ of the total 4GB address space. To solve this issue, use the Register Addressing \texttt{jr} and \texttt{jalr} commands. Because these use full 32-bit register values, the entire address space can be jumped to with these commands. 

As stated in section \ref{PC-Relative Addressing}, the addresses used for jump addressing are computed during the linking phase of machine code generation, which can be slower than other addressing methods. 


%% No problem ideas yet, but come up with some!
\section{Solutions}

\ref{RType1}: The binary for this operation is $000000 00010 00011 00001 00000 100000$. \\ OP = $000000$ \\ RS = $00010$ \\ RT = $00011$ \\ RD = $00001$ \\ SHAMT = $00000$ \\ FUNC = $100000$

\ref{IType1}: This translates to the instruction \texttt{bne \$t0 \$t1 0xaaaa}. \\ BNE = $000101$ \\ \$t0 = $01000$ \\ \$t1 = $01001$ \\ $0xaaaa$ = $1010101010101010$

\printindex

\end{document}
