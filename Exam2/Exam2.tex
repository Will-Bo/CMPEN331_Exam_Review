\documentclass{article}

\usepackage{amsmath, amsfonts, amssymb}

\title{CMPEN 331 Exam 2 Review}
\author{Will Bochnowicz}

\usepackage{imakeidx}
%\makeindex
\makeindex[columns=2, title=Index, intoc]

\begin{document}

\maketitle

\tableofcontents

\section{Floating Point}\label{Floating Point}\index{Floating Point}
%% What is is, description of the standard

The standard of floating point numbers are described in IEEE 754\index{Floating Point!Standard}. This was created because of a divergence in early computing where different manufacturers were using different means of storing floating point numbers, making program portability more difficult for developers. 

\subsection{Floating Point Representation}\index{Floating Point!Representation}
%% Diagram of representation, number of bits, mantissa, how to get to decimal, bias calculation
The representation of a floating point number is split into three parts:

\begin{tabular}{|c|c|c|c|}
    \hline
     Precision & Sign Bit & Exponent & Fraction \\
     \hline
     Single & 1 bit & 8 bits & 23 bits \\
     \hline
     Double & 1 bit & 11 bits & 52 bits \\
     \hline
\end{tabular}

Where the number represented is calculated as $(-1)^S \times (1 + F) \times 2^{E-B}$\index{Floating Point!Equation}, $B$ stands for Bias. The bias \index{Bias} is used to allow representations of both positive and negative exponent values. In order to calculate it, use the following equation: $2^{(\text{Exponent Bit Count}-1)}-1$.

This system is not perfect. When doing operations on floating point numbers, it is possible to have errors with your handling of the exponent field. Overflow\index{Overflow} is when the exponent field after an operation is too great to fit in the allotted number of bits, while underflow\index{Underflow} occurs when the negative exponent is too large to fit in the exponent field. 

\subsubsection{Floating Point Problems}\label{Floating Point Problems}\index{Floating Point Problems}

\begin{equation}\label{Problem1}
    \text{How would you represent the value } 4.125 \text{ in binary?}
\end{equation}

\begin{equation}\label{Problem2}
    \text{What is the following single-precision float in decimal? } \\ 
    1 01011010 01100011011000110110001
\end{equation}

\subsection{Floating Point Operations}\label{Floating Point Operations}\index{Floating Point!Operations}
%% How addition and multiplication work, short description of hardware. Maybe pull screenshots from zyBook. 

\section{MIPS Datapath Design}

\subsection{Single-Cycle}

\subsection{Pipelining}

\section{Solutions}

\ref{Problem1}: 

\ref{Problem2}:

\printindex
\end{document}
