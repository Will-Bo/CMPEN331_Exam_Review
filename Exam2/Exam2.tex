\documentclass{article}

\usepackage{amsmath, amsfonts}

\title{CMPEN 331 Exam 2 Review}
\author{Will Bochnowicz}

\begin{document}

\maketitle

\section{Floating Point}
%% What is is, description of the standard

\subsection{Floating Point Representation}
%% Diagram of representation, number of bits, mantissa, how to get to decimal, bias calculation

\begin{equation}\label{Problem1}
    \text{How would you represent the value } 4.125 \text{ in binary?}
\end{equation}

\begin{equation}\label{Problem2}
    \text{What is the following single-precision float in decimal? } \\ 
    1 01011010 01100011011000110110001
\end{equation}

\subsection{Floating Point Operations}
%% How addition and multiplication work, short description of hardware. Maybe pull screenshots from zyBook. 

\section{Solutions}

\ref{Problem1}: 

\ref{Problem2}:
\end{document}
